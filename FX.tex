\documentclass{article}
\usepackage{amsmath}
\usepackage{amsthm}
\usepackage{enumerate}
\usepackage{subfigure}
\usepackage[bookmarks=true]{hyperref}
\usepackage{bookmark}
\usepackage{graphicx}

\input{../work/physics_common}
\title{On Foreign Exchange Rates}
\author{Xie Xiaolei}
\date{\today}
\begin{document}
\maketitle

\section{The eigenvector of the smallest eigenvalue}\label{sec:acf}
Construct a portfolio according to the eigenvector associated with the
smallest eigenvalue of the covariance matrix. Call it the
least-variance portfolio. The auto-correlation function of the
least-variance portfolio is shown in figure \ref{fig:1-day-ret-acf}:
\begin{figure}[htb!]
  \centering
  \includegraphics[scale=0.3]{least-variance-ptfl-acf.pdf}
  \caption{Auto-correlation function of the least-variance
    portfolio. The portfolio is constructed according to the
    eigenvector associated with the smallest eigenvalue of the
    covariance matrix of 1-day returns of 22 foreign exchange rates
    against the Swedish crown.}
  \label{fig:1-day-ret-acf}
\end{figure}

The eigenvector of the smallest eigenvalue has components as plotted
in figure \ref{fig:V_p-components}.
\begin{figure}[htb!]
  \centering
  \includegraphics[scale=0.3]{eigenvector_p_comp.pdf}
  \caption{Components of the eigenvector associated with the smallest
    eigenvalue.}
  \label{fig:V_p-components}
\end{figure}

\section{The largest eigenvalues}
The eigenvalues of the sample covariance matrix is plotted in figure
\ref{fig:Eigenvalues_of_FX_cov}, and the components of the
eigenvectors associated with the largest 6 eigenvalues are plotted in
figure \ref{fig:eigenvector_components_largest_values1} and
\ref{fig:eigenvector_components_largest_values2}.

\begin{figure}[htb!]
  \centering
  \includegraphics[scale=0.3]{Eigenvalues_of_FX_cov.pdf}
  \caption{Eigenvalues of the sample covariance matrix.}
  \label{fig:Eigenvalues_of_FX_cov}
\end{figure}


\begin{figure}[htb!]
  \centering
  \subfigure[FX Covariance $V_1$]{
    \includegraphics[scale=0.3]{eigenvector_1_comp.pdf}
  }
  \subfigure[SP500 Covariance $V_1$]{
    \includegraphics[scale=0.3]{../work/papers/segments/V1_CovarianceMatrix.pdf}
  }
  \subfigure[FX Covariance $V_2$]{
    \includegraphics[scale=0.3]{eigenvector_2_comp.pdf}
  }
  \subfigure[SP500 Covariance $V_2$]{
    \includegraphics[scale=0.3]{../work/papers/segments/V2_CovarianceMatrix.pdf}
  }
  \subfigure[FX Covariance $V_3$]{
    \includegraphics[scale=0.3]{eigenvector_3_comp.pdf}
  }
  \subfigure[SP500 Covariance $V_3$]{
    \includegraphics[scale=0.3]{../work/papers/segments/V3_CovarianceMatrix.pdf}
  }
  \caption{Eigenvectors corresponding to the largest eigenvalues}
  \label{fig:eigenvector_components_largest_values1}
\end{figure}

\begin{figure}[htb!]
  \centering
  \subfigure[FX Covariance $V_4$]{
    \includegraphics[scale=0.3]{eigenvector_4_comp.pdf}
  }
  \subfigure[SP500 Covariance $V_4$]{
    \includegraphics[scale=0.3]{../work/papers/segments/V4_CovarianceMatrix.pdf}
  }
  \subfigure[FX Covariance $V_5$]{
    \includegraphics[scale=0.3]{eigenvector_5_comp.pdf}
  }
  \subfigure[SP500 Covariance $V_5$]{
    \includegraphics[scale=0.3]{../work/papers/segments/V5_CovarianceMatrix.pdf}
  }
  \subfigure[FX Covariance $V_6$]{
    \includegraphics[scale=0.3]{eigenvector_6_comp.pdf}
  }
  \subfigure[SP500 Covariance $V_6$]{
    \includegraphics[scale=0.3]{../work/papers/segments/V6_CovarianceMatrix.pdf}
  }
  \caption{Eigenvectors corresponding to the largest eigenvalues}
  \label{fig:eigenvector_components_largest_values2}
\end{figure}


\bibliographystyle{unsrt}
\bibliography{../work/thesis/econophysics}
\end{document}
