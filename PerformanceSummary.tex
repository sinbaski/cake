\title{Performance of Xie's Trading Algorithm}
\author{
  Xie Xiaolei \\
  xie.xiaolei@gmail.com, +44 73 80 19 11 66
}
\date{\today}

\documentclass[12pt]{article}
\usepackage{multirow}
\usepackage{graphicx}

\DeclareGraphicsExtensions{.pdf,.png,.jpg}

\input{../kkasi/physics_common.tex}
% \usepackage{amsmath}
% \usepackage{amsfonts}
% \usepackage{mathrsfs}
\usepackage[colorlinks,citecolor=blue,urlcolor=blue]{hyperref}
% \usepackage[bookmarks=true]{hyperref}
% \usepackage{bookmark}
% \usepackage{dsfont}
% \usepackage{enumerate}

% \newcommand{\E}{
%         \mathbb{E}
% }
% \newcommand{\p}{
%         \mathbb{P}
% }
% \newcommandf166{\I}[1]{
%         \mathbf 1_{#1}
% }
% \newcommand{\dom}{
%         \text{dom}
% }
% \newcommand{\inn}[1]{
%   \langle#1\rangle
% }

\begin{document}
\maketitle

\section{Introduction}
The algorithm is profitable trading ETFs that provide exposure to a
selection of foreign exchange rates and agricultural commodities. In
the following sections we show the performance of the algorithm when
trading these two classes of ETFs. A return in the following refers to
the return given by close-to-close prices unless stated otherwise
explicitly.

The algorithm is built on principle component analysis, ARIMA
modelling of time series and volatility proxies. Resampling techniques
are also essential.

The ETFs are chosen according to the tail indices of their daily
returns' distributions as well as to their cross correlations. See
\S\ref{sec:TailIndices} for more details.

\section{Foreign Exchange Rates}
The strategy is tested on the trading data of UUP (US dollar), FXE
(Euro) and FXY (Japanese Yen) in the period 24th Auguest, 2012 to 19th
March, 2018, covering 1399 trading days. The strategy {\bf uses and
  only uses daily high, low and close prices}. Note that the Brexit
referendum was held on 23rd June, 2016 and had caused significant
disruptions to the FX market.

Table \ref{tab:fx1} shows the performance of the strategy during this
period.
\begin{table}[htb!]
  \centering
  \begin{tabular}{ll||ll}
    \multicolumn{2}{c||}{up to 19 March, 2018}
    & \multicolumn{2}{c}{up to 23 June, 2016} \\
    \hline
    $R_D$ & $2.63 \times 10^{-4}$ & $R_D$ & $4.26 \times 10^{-4}$  \\
    $S_D$ & 0.084 & $S_D$ & 0.130 \\
    $R_M$ & $3.70 \times 10^{-3}$ & $R_M$ & $6.39 \times 10^{-3}$  \\
    $S_M$ & 0.294 & $S_M$ & 0.463 \\
    $D_{\text{max}}$ & 9.8\% & $D_{\text{max}}$ & 6.9\% \\
    $\bar X$ & 78.45\% & $\bar X$ & 80.92\% \\
    $\bar Y$ & 88.69\% & $\bar Y$ & 89.04\% \\
    profit & 26.0\% & profit & 28.3\%
  \end{tabular}
  \caption{
    Performance of the algorithm on FX rates.
    $R_D$: expected value of daily returns;
    $S_D$: Sharpe ratio of daily returns;
    $R_D$: expected value of monthly returns;
    $S_M$: Sharpe ratio of monthly returns.
    $D_\M$: maximum drawdown for the test period. Monthly
    returns are computed using monthly average worth of the
    portfolio.
    $\bar X$: average exposure. The exposure of day $n$ is
    $X_n = (L_n + S_n)/W_n$, where $L_n$ ($S_n$) is the absolute value
    of the total worth of long (short) positions on market close of
    day $n$; $W_n$ is the worth of the portfolio on day $n$. $\bar X$
    is the average of $\{X_n\}$ over the test period.
    $\bar Y$: average turnover ratio. Turnover ratio of day $n$ is
    computed as $\frac{1}{W_{n-1}}\sum_{i=1}^3 \left| x_{i, n} - x_{i,
        n-1} \right|$, where $x_{i, n}$ is the monetary amount of
    allocation to asset $i$ on day $n$.
  }
  \label{tab:fx1}
\end{table}
The distribution of daily returns (denoted $X$) during this period is
not normal but has heavy tails. Specifically,
\begin{eqnarray*}
  \P(X > x) &\sim& L_r(x) x^{-\alpha_r}, \quad x \to \infty \\
  \P(X \leq -x) &\sim& L_l(x) x^{-\alpha_l}, \quad x \to \infty
\end{eqnarray*}
where $L_r$ and $L_l$ are slowly varying functions that satisfy
\[
  \lim_{x \to \infty} {L(a x) \over L(x)} = 1, \quad \forall a > 0.
\]
Here $\sim$ denotes tail equivalence, i.e. $f(x) \sim g(x)$ means, for
two functions $f(x)$ and $g(x)$, $x > 0$,
\[
  \lim_{x \to \infty} {f(x) \over g(x)} = 1
\]
See Embrechts, Kl\"ueppelberg and Mikosch
\cite{embrechts:klueppelberg:mikosch:1997} for more details on
heavy-tailed distributions and {\it Extreme Value Theory}. The tail
index $\alpha_r$ ($\alpha_l$) gives an insight into the likelihood
of large positive (negative) returns. {\bf If $\alpha_r < \alpha_l$,
  then for a sufficiently large positive $u$, $\P(X > u) > \P(X \leq -u)$},
i.e. large profits are more likely than equally large losses,
assuming the same slowly varying function is shared by both
tails. Using the Hill estimator, the tail indices $\alpha_r$ and
$\alpha_l$ are estimated for the testing period:
\begin{eqnarray*}
  \alpha_r &=& 3.61, \quad \text{for profits} \\
  \alpha_l &=& 4.39, \quad \text{for losses}
\end{eqnarray*}
Meanwhile, the sample mean of the daily returns during the entire
period is estimated at $1.77 \times 10^{-4}$ and a t-test of $\E X =
0$ against $\E X \neq 0$ gives a p-value of $9.78 \times 10^{-3}$.
So we see that the algorithm succeeds in predicting price movements,
especially in predicting large price movements, considerably more
often than it fails.

From the perspective of predictive power, some more insight about the
algorithm can be obtained by investigating the correlation between an
asset's return on day $n$ and the algorithm's position on the same
asset on day $n-1$.

Let $X_{i, t}$ denote the return of asset $i, i=1,2,3$ on day $t,
t=1,2, \dots$ (that is, $X_{i,t} = P_{i,t}/P_{i, t-1} - 1$ where
$P_{i,t}$ is the price of asset $i$ on market close of day $t$) and $Q_{i, t}$
denote the size of the algorithm's position on asset $i$ on day
$t$. $-1 \leq Q_{i,t} \leq 1$. When $Q_{i,t} = -1$, the algorithm
invests all available capital on a short position of asset $i$; when
$Q_{i,t} = 1$, its invests all available capital on a long position of
asset $i$. We compute the Pearson correlation coefficient $\rho$ and
Kendall's $\tau$ of the following two series:
\begin{eqnarray*}
  x &=& (Q_{1, 1}, \dots, Q_{1, n-1}, Q_{2, 1}, \dots, Q_{2, n-1}, Q_{3,
        1}, \dots, Q_{3, n-1}) \\
  y &=& (X_{1, 2}, \dots, X_{1, n}, X_{2, 2}, \dots, X_{2, n}, X_{3,
        2}, \dots, X_{3, n})
\end{eqnarray*}
where $n$ is the number of trading days in the testing period. In this
calculation we exclude days on which the algorithm decides not to
trade. For the 5.5-year testing period, $n=1123$. The resulting
correlation coefficients are
\begin{eqnarray*}
  \text{Pearson's } \rho &=& 0.030 \\
  \text{Kendall's } \tau &=& 0.018
\end{eqnarray*}

% Figure \ref{fig:fx1} shows the worth of a hypothetic FX portfolio that
% starts on 24th August, 2012 and that is worth 1 unit of currency at
% that time.
% \begin{figure}[htb!]
%   \centering
%   \includegraphics[width=1.0\linewidth]{FX.pdf}
%   \caption{FX portfolio worth}
%   \label{fig:fx1}
% \end{figure}
% Table \ref{tab:fx2} shows the average worth of the portfolio in each
% quarter during the test period.
% \begin{table}[htb!]
%   \centering
%   \begin{tabular}{ccc||ccc}
%     year & quarter & worth & year & quarter & worth \\
%     \hline
%     2012 & 3 & 1.006 & 2015 & 3 & 1.352 \\
%     2012 & 4 & 1.004 & 2015 & 4 & 1.352 \\
%     2013 & 1 & 1.037 & 2016 & 1 & 1.347 \\
%     2013 & 2 & 1.069 & 2016 & 2 & 1.324 \\
%     2013 & 3 & 1.073 & 2016 & 3 & 1.298 \\
%     2013 & 4 & 1.109 & 2016 & 4 & 1.282 \\
%     2014 & 1 & 1.105 & 2017 & 1 & 1.293 \\
%     2014 & 2 & 1.090 & 2017 & 2 & 1.309 \\
%     2014 & 3 & 1.120 & 2017 & 3 & 1.308 \\
%     2014 & 4 & 1.231 & 2017 & 4 & 1.285 \\
%     2015 & 1 & 1.320 & 2018 & 1 & 1.286 \\
%     2015 & 2 & 1.364 & & &
%   \end{tabular}
%   \caption{Quarterly average worth of the FX portfolio that starts on
%     24th Auguest, 2012 with worth 1.
%   }
%   \label{tab:fx2}
% \end{table}
\section{Agricultural Commodities}
The same strategy also proves profitable when tested on agricultural
commodities, namely JJG (ETN based on grains futures), WEAT (commodity pool
based on wheat futures) and SOYB (commodity pool based on soybean
futures). Table \ref{tab:agricomm1} gives a summary of the performance
of the algorithm during the period 24th Augest, 2012 to 2nd April,
2018.
\begin{table}[htb!]
  \centering
  \begin{tabular}{ll||ll}
    \multicolumn{2}{c||}{up to 2 April, 2018}
    & \multicolumn{2}{c}{up to 23 June, 2016} \\
    \hline
    $R_D$ & $1.77 \times 10^{-4}$ & $R_D$ & $2.42 \times 10^{-4}$  \\
    $S_D$ & 0.069 & $S_D$ & 0.091 \\
    $R_M$ & $3.00 \times 10^{-3}$ & $R_M$ & $4.49 \times 10^{-3}$  \\
    $S_M$ & 0.273 & $S_M$ & 0.368 \\
    $D_{\text{max}}$ & 7.0\% & $D_{\text{max}}$ & 7.0\% \\
    $\bar X$ & 53.49\% & $\bar X$ & 50.32\% \\
    $\bar Y$ & 62.47\% & $\bar Y$ & 58.07\% \\
    profit & 21.94\% & profit & 20.65\% 
  \end{tabular}
  \caption{
    Performance of the algorithm on agricultural commodities.
    $R_D$: expected value of daily returns;
    $S_D$: Sharpe ratio of daily returns;
    $R_D$: expected value of monthly returns;
    $S_M$: Sharpe ratio of monthly returns.
    $D_\M$: maximum drawdown for the test period. Monthly
    returns are computed using monthly average worth of the
    portfolio.
    $\bar X$: average exposure. The exposure of day $n$ is
    $X_n = (L_n + S_n)/W_n$, where $L_n$ ($S_n$) is the absolute value
    of the total worth of long (short) positions on market close of
    day $n$; $W_n$ is the worth of the portfolio on day $n$. $\bar X$
    is the average of $\{X_n\}$ over the test period.
    $\bar Y$: average turnover ratio. Turnover ratio of day $n$ is
    computed as $\frac{1}{W_{n-1}}\sum_{i=1}^3 \left| x_{i, n} - x_{i,
        n-1} \right|$, where $x_{i, n}$ is the monetary amount of
    allocation to asset $i$ on day $n$.
  }
  \label{tab:agricomm1}
\end{table}

In the same way as we have done for the FX ETFs, we also estimate the
tail indices of the distribution of the daily returns generated by the
algorithm on the aforementioned products. The results are as follows:
\begin{eqnarray*}
  \alpha_r &=& 3.58 \\
  \alpha_l &=& 3.73
\end{eqnarray*}
As a reminder of the meaning of these two quantities, $\P(X > x) \sim
L(x) x^{-\alpha_r}$ and $\P(X \leq x) \sim L(x) x^{-\alpha_l}$, where
$X$ has the same distribution as the daily returns generated by the
algorithm on those days when trading occurs.

Using the same method as described for FX ETFs, the correlation
between next-day returns and present-day positions is also estimated:
\begin{eqnarray*}
  \text{Pearson's } \rho &=& 0.013 \\
  \text{Kendall's } \tau &=& 0.005
\end{eqnarray*}

\section{Tail Indices \& Correlations}\label{sec:TailIndices}
The algorithm is built on ARIMA models and hence one can expect it to
perform better with symbols whose returns series have lighter
tails. It is a stylized fact that financial return series have heavy
tails, i.e. for $x > 0$, $\P(X > x)\sim L(x) x^{-\alpha_r}$ and
$\P(X \leq -x)\sim L(x) x^{-\alpha_l}$, where we assume each element
of the return series, call it $\{X_t\}_{t=0,1,2,\dots}$, are
independent and identically distributed (iid). The tail indices give a
measure of how heavy the tails are: the smaller the tail index, the
heavier the tail. I expect the algorithm to perform better on symbols
with larger tail indices and hence have estimated the tail indices of
poptentially suitable ETFs. The result is listed in table
\ref{tab:TailIndices}.
\begin{table}[htb!]
  \centering
  \begin{tabular}{r|c|c||r|c|c||r|c|c}
    symbol & $\alpha_l$ & $\alpha_r$
    & symbol & $\alpha_l$ & $\alpha_r$
    & symbol & $\alpha_l$ & $\alpha_r$ \\
    \hline
    UUP & 3.49 & 3.64 
    & JJG & 3.05 & 4.20 
    & SPY & 2.48 & 2.41 \\
    FXE & 3.95 & 3.30 
    & WEAT & 4.80 & 3.01
    & EWG & 2.72 & 2.73 \\
    FXY & 3.18 & 3.09 
    & SOYB & 2.82 & 3.77 
    & EWJ & 2.50 & 2.28
  \end{tabular}
  \caption{Tail indices}
  \label{tab:TailIndices}
\end{table}
So we see that the FX rates and the agricultural commodities have lighter
tails than do country ETFs such as SPY (tracking {\it US} stocks), EWG
(tracking {\it German} stocks) and EWJ (tracking {\it Japanese}
stocks).

Another criterion for selecting symbols to the algorithm is strong
cross correlations. The Pearson correlation coefficients of the
FX rates and of the agricultural commodities in the testing period are
tabulated below in table \ref{tab:cor1}.
\begin{table}[htb!]
  \begin{minipage}[t]{0.5\linewidth}
    \centering
    \begin{tabular}{c|c|c|c}
      & UUP & FXE & FXY \\
      \hline
      UUP & 1.0 & -0.955 & -0.346 \\
      FXE & -0.955 & 1.0 & 0.216 \\
      FXY & -0.346 & 0.216 & 1.0
    \end{tabular}
  \end{minipage}\hfill
  \begin{minipage}[t]{0.5\linewidth}
    % \begin{table}[htb!]
    \centering
    \begin{tabular}{c|c|c|c}
      & JJG & WEAT & SOYB \\
      \hline
      JJG & 1.0 & 0.752 & 0.697 \\
      WEAT & 0.752 & 1.0 & 0.391 \\
      SOYB & 0.697 & 0.391 & 1.0
    \end{tabular}
  \end{minipage}
  \caption{Correlation coefficients of FX and agricultural commodities}
  \label{tab:cor1}
\end{table}
In comparison, the coefficients of the country ETFs are tabulated in
table \ref{tab:cor2}.
\begin{table}[htb!]
  \centering
  \begin{tabular}{c|c|c|c}
    & SPY & EWG & EWJ \\
    \hline
    SPY & 1.0 & 0.857 & 0.796 \\
    EWG & 0.857 & 1.0 & 0.750 \\
    EWJ & 0.796 & 0.750 & 1.0
  \end{tabular}
  \caption{Correlation coefficients of country ETFs}
  \label{tab:cor2}
\end{table}
While the country ETFs are seen to have stronger correlations than do
the corresponding currencies, their tails are significantly heavier
than are those of the currencies. I give precedence to lighter tails,
because by the {\it Marcinkiewicz-Zygmund strong law of large
  numbers}, estimating the true correlation coefficient using its
sample counterpart is less accurate when the series in question have
heavier tails.

Obviously, tail indices and correlation coefficients are themselves
varying in time, but such variations are much slower than the
variations of the returns series that these quantities describe. For
the purpose of choosing assets to trade with the algorithm, estimation
of these quantities using a long series of historical data should be
sufficiently accurate.

\bibliographystyle{unsrt}
\bibliography{../kkasi/thesis/econophysics}
\end{document}
