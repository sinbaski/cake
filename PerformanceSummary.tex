\title{Summary of Xie's Trading Algorithm}
\author{
  Xie Xiaolei \\
  xie.xiaolei@gmail.com, +44 73 80 19 11 66
}
\date{\today}

\documentclass[12pt]{article}
\usepackage{multirow}
\usepackage{graphicx}

\DeclareGraphicsExtensions{.pdf,.png,.jpg}

\input{../kkasi/physics_common.tex}
% \usepackage{amsmath}
% \usepackage{amsfonts}
% \usepackage{mathrsfs}
\usepackage[colorlinks,citecolor=blue,urlcolor=blue]{hyperref}
% \usepackage[bookmarks=true]{hyperref}
% \usepackage{bookmark}
% \usepackage{dsfont}
% \usepackage{enumerate}

% \newcommand{\E}{
%         \mathbb{E}
% }
% \newcommand{\p}{
%         \mathbb{P}
% }
% \newcommandf166{\I}[1]{
%         \mathbf 1_{#1}
% }
% \newcommand{\dom}{
%         \text{dom}
% }
% \newcommand{\inn}[1]{
%   \langle#1\rangle
% }

\begin{document}
\maketitle

\section{Introduction}
The algorithm is profitable trading ETFs that provide exposure to a
selection of foreign exchange rates and agricultural commodities. In
the following sections we show the performance of the algorithm when
trading these two classes of ETFs. A return in the following refers to
the return given by close-to-close prices unless stated otherwise
explicitly.

The algorithm is built on principle component analysis, ARIMA
modelling of time series and volatility proxies. Resampling techniques
are also essential.

The ETFs are chosen according to the tail indices of their returns'
distributions. See \S\ref{sec:TailIndices} for more details.

\section{Foreign Exchange Rates}
The strategy is tested on the trading data of UUP (US dollar), FXE
(Euro) and FXY (Japanese Yen) in the period 24th Auguest, 2012 to 19th
March, 2018, covering 1399 trading days. The strategy {\bf uses and
  only uses daily high, low and close prices}. Note that the Brexit
referendum was held on 23rd June, 2016 and had caused significant
disruptions to the FX market.

Table \ref{tab:fx1} shows the performance of the strategy during this
period.
\begin{table}[htb!]
  \centering
  \begin{tabular}{ll||ll}
    \multicolumn{2}{c||}{up to 19 March, 2018}
    & \multicolumn{2}{c}{up to 23 June, 2016} \\
    \hline
    $S_D$ & 0.084 & $S_D$ & 0.130 \\
    $S_M$ & 0.294 & $S_M$ & 0.463 \\
    $D_{\text{max}}$ & 9.8\% & $D_{\text{max}}$ & 6.9\% \\
    $\bar X$ & 78.45\% & $\bar X$ & 80.92\% \\
  \end{tabular}
  \caption{Performance of FX strategy. $S_D$: Sharpe ratio of daily
    returns; $S_M$: Sharpe ratio of monthly returns. $D_\M$:
    maximum drawdown for the test period. Monthly returns are computed
    using monthly average worth of the portfolio. $\bar X$: average
    exposure. The exposure of day $n$ is $X_n = (L_n + S_n)/W_n$,
    where $L_n$ ($S_n$) is the absolute value of the total worth of
    long (short) positions on market close of day $n$; $W_n$ is the
    worth of the portfolio on day $n$. $\bar X$ is the average of
    $\{X_n\}$ over the test period.
  }
  \label{tab:fx1}
\end{table}
The distribution of daily returns (denoted $X$) during this period is
not normal but has heavy tails. Specifically,
\begin{eqnarray*}
  \P(X > x) &\sim& L_r(x) x^{-\alpha_r}, \quad x \to \infty \\
  \P(X \leq -x) &\sim& L_l(x) x^{-\alpha_l}, \quad x \to \infty
\end{eqnarray*}
where $L_r$ and $L_l$ are slowly varying functions that satisfy
\[
  \lim_{x \to \infty} {L(a x) \over L(x)} = 1, \quad \forall a > 0.
\]
Here $\sim$ denotes tail equivalence, i.e. $f(x) \sim g(x)$ means, for
two functions $f(x)$ and $g(x)$, $x > 0$,
\[
  \lim_{x \to \infty} {f(x) \over g(x)} = 1
\]
See Embrechts, Kl\"ueppelberg and Mikosch
\cite{embrechts:klueppelberg:mikosch:1997} for more details on
heavy-tailed distributions and {\it Extreme Value Theory}. The tail
index $\alpha_r$ ($\alpha_l$) gives an insight into the likelihood
of large positive (negative) returns. {\bf If $\alpha_r < \alpha_l$,
  then for a sufficiently large positive $u$, $\P(X > u) > \P(X \leq -u)$},
i.e. large profits are more likely than equally large losses,
assuming the same slowly varying function is shared by both
tails. Using the Hill estimator, the tail indices $\alpha_r$ and
$\alpha_l$ are estimated for the testing period:
\begin{eqnarray*}
  \alpha_r &=& 3.61, \quad \text{for profits} \\
  \alpha_l &=& 4.39, \quad \text{for losses}
\end{eqnarray*}
So we see that the algorithm succeeds in predicting large price swings
considerably more often than it fails. From the perspective of
predictive power, some more insight about the algorithm can be
obtained by investigating the correlation between an asset's return
on day $n$ and the algorithm's position on the same asset on day
$n-1$.

Let $X_{i, t}$ denote the return of asset $i, i=1,2,3$ on day $t,
t=1,2, \dots$ (that is, $X_{i,t} = P_{i,t}/P_{i, t-1} - 1$ where
$P_{i,t}$ is the price of asset $i$ on market close of day $t$) and $Q_{i, t}$
denote the size of the algorithm's position on asset $i$ on day
$t$. $-1 \leq Q_{i,t} \leq 1$. When $Q_{i,t} = -1$, the algorithm
invests all available capital on a short position of asset $i$; when
$Q_{i,t} = 1$, its invests all available capital on a long position of
asset $i$. We compute the Pearson correlation coefficient $\rho$ and
Kendall's $\tau$ of the following two series:
\begin{eqnarray*}
  x &=& (Q_{1, 1}, \dots, Q_{1, n-1}, Q_{2, 1}, \dots, Q_{2, n-1}, Q_{3,
        1}, \dots, Q_{3, n-1}) \\
  y &=& (X_{1, 2}, \dots, X_{1, n}, X_{2, 2}, \dots, X_{2, n}, X_{3,
        2}, \dots, X_{3, n})
\end{eqnarray*}
where $n$ is the number of trading days in the testing period. In this
calculation we exclude days on which the algorithm decides not to
trade. For the 5.5-year testing period, $n=1123$. The resulting
correlation coefficients are
\begin{eqnarray*}
  \text{Pearson's } \rho &=& 0.030 \\
  \text{Kendall's } \tau &=& 0.018
\end{eqnarray*}

% Figure \ref{fig:fx1} shows the worth of a hypothetic FX portfolio that
% starts on 24th August, 2012 and that is worth 1 unit of currency at
% that time.
% \begin{figure}[htb!]
%   \centering
%   \includegraphics[width=1.0\linewidth]{FX.pdf}
%   \caption{FX portfolio worth}
%   \label{fig:fx1}
% \end{figure}
% Table \ref{tab:fx2} shows the average worth of the portfolio in each
% quarter during the test period.
% \begin{table}[htb!]
%   \centering
%   \begin{tabular}{ccc||ccc}
%     year & quarter & worth & year & quarter & worth \\
%     \hline
%     2012 & 3 & 1.006 & 2015 & 3 & 1.352 \\
%     2012 & 4 & 1.004 & 2015 & 4 & 1.352 \\
%     2013 & 1 & 1.037 & 2016 & 1 & 1.347 \\
%     2013 & 2 & 1.069 & 2016 & 2 & 1.324 \\
%     2013 & 3 & 1.073 & 2016 & 3 & 1.298 \\
%     2013 & 4 & 1.109 & 2016 & 4 & 1.282 \\
%     2014 & 1 & 1.105 & 2017 & 1 & 1.293 \\
%     2014 & 2 & 1.090 & 2017 & 2 & 1.309 \\
%     2014 & 3 & 1.120 & 2017 & 3 & 1.308 \\
%     2014 & 4 & 1.231 & 2017 & 4 & 1.285 \\
%     2015 & 1 & 1.320 & 2018 & 1 & 1.286 \\
%     2015 & 2 & 1.364 & & &
%   \end{tabular}
%   \caption{Quarterly average worth of the FX portfolio that starts on
%     24th Auguest, 2012 with worth 1.
%   }
%   \label{tab:fx2}
% \end{table}
\section{Agricultural Commodities}
The same strategy also proves profitable when tested on agricultural
commodities, namely JJG (ETN based on grains futures), WEAT (commodity pool
based on wheat futures) and SOYB (commodity pool based on soybean
futures). Table \ref{tab:agricomm1} gives a summary of the performance
of the algorithm during the period 24th Augest, 2012 to 2nd April,
2018.
\begin{table}[htb!]
  \centering
  \begin{tabular}{ll||ll}
    \multicolumn{2}{c||}{up to 19 March, 2018}
    & \multicolumn{2}{c}{up to 23 June, 2016} \\
    \hline
    $S_D$ & 0.069 & $S_D$ & 0.091 \\
    $S_M$ & 0.273 & $S_M$ & 0.368 \\
    $D_{\text{max}}$ & 7.0\% & $D_{\text{max}}$ & 7.0\% \\
    $\bar X$ & 53.49\% & $\bar X$ & 50.32\% \\
  \end{tabular}
  \caption{Performance of FX strategy. $S_D$: Sharpe ratio of daily
    returns; $S_M$: Sharpe ratio of monthly returns. $D_\M$:
    maximum drawdown for the test period. Monthly returns are computed
    using monthly average worth of the portfolio. $\bar X$: average
    exposure. The exposure of day $n$ is $X_n = (L_n + S_n)/W_n$,
    where $L_n$ ($S_n$) is the absolute value of the total worth of
    long (short) positions on market close of day $n$; $W_n$ is the
    worth of the portfolio on day $n$. $\bar X$ is the average of
    $\{X_n\}$ over the test period.
  }
  \label{tab:agricomm1}
\end{table}



\section{Tail Indices}\label{sec:TailIndices}

\bibliographystyle{unsrt}
\bibliography{../kkasi/thesis/econophysics}
\end{document}
