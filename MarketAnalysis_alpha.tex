\documentclass{article}

\RequirePackage[OT1]{fontenc}
\RequirePackage{amsthm,amsmath}
\RequirePackage[numbers]{natbib}
\RequirePackage[colorlinks,citecolor=blue,urlcolor=blue]{hyperref}

\usepackage{graphicx}
\usepackage{algorithm}
\usepackage{algpseudocode}

\input{../work/physics_common.tex}
\title{Market Analysis and Alpha Generation}
\author{Xie Xiaolei}
\date{\today}
\begin{document}
\maketitle

\section{A single asset}
Let $P_t$ denote the closing prices of an asset, $\overline P_t$ the
highest price, $\underline P_t$ the lowest price and $V_t$ the volume
or turnover of the same asset. These are the data directly
available. Define
\begin{eqnarray*}
  X_t &=& \ln P_{t} - \ln P_{t-1} \\
  \overline X_t &=& \ln \overline P_{t} - \ln \overline P_{t-1} \\
  \underline X_t &=& \ln \underline P_{t} - \ln \underline P_{t-1} \\
  W_t &=& \ln V_t - \ln V_{t-1}
\end{eqnarray*}

\subsection{S\&P 500 and DJIA}
There are significant correlations between $X_t$ and $W_t$, as is
illustrated below for S\&P 500 and Dow Jones:
\begin{figure}[!htb]
\begin{minipage}{0.5\linewidth}
  \includegraphics[width=\textwidth]{SP500_CloseVolume_ccf.pdf}
\end{minipage}\hfill
\begin{minipage}{0.5\linewidth}
  \includegraphics[width=\textwidth]{DJIA_CloseVolume_ccf.pdf}
\end{minipage}
\caption{Return-Turnover correlation of S\&P 500 and DJIA. Data of 25
  weeks (120 observations) ending on 2015-03-07 are used.}
\end{figure}

Siginificant correlations also exist between $X_t$ and $\overline X_t$
as well as between $X_t$ and $\underline X_t$:
\begin{figure}[!htb]
\begin{minipage}{0.33\linewidth}
  \includegraphics[width=\textwidth]{SP500_CloseHigh_ccf.pdf}
\end{minipage}\hfill
\begin{minipage}{0.33\linewidth}
  \includegraphics[width=\textwidth]{SP500_CloseLow_ccf.pdf}
\end{minipage}\hfill
\begin{minipage}{0.33\linewidth}
  \includegraphics[width=\textwidth]{SP500_HighLow_ccf.pdf}
\end{minipage}
\begin{minipage}{0.33\linewidth}
  \includegraphics[width=\textwidth]{DJIA_CloseHigh_ccf.pdf}
\end{minipage}\hfill
\begin{minipage}{0.33\linewidth}
  \includegraphics[width=\textwidth]{DJIA_CloseLow_ccf.pdf}
\end{minipage}\hfill
\begin{minipage}{0.33\linewidth}
  \includegraphics[width=\textwidth]{DJIA_HighLow_ccf.pdf}
\end{minipage}
\caption{Close-High and Close-Low correlation of S\&P 500 and
  DJIA. Data of 25 weeks (120 observations) ending on 2015-03-07 are
  used.}
\end{figure}

The figures prompt a VARMA model to describe the data. Let
\[
\vec Y_t = (\overline X_t, X_t, \underline X_t, W_t)
\]
Then the model is
\[
\vec Y_t = \sum_{i=1}^p \Phi^i \vec Y_{t-i} + \sum_{j=1}^q \Theta^j \vec Z_{t-j}
\]
where each $\Phi^i$ and $\Theta^j$ is a $4 \times 4$ matrix. Their
entries can be determined using e.g. the MTS package of R. The
parameters are estimated using conditional maximum likelihood methods.
\end{document}
